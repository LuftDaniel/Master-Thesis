

%%%%%%%%%%%%%%%%%%%%%%%%%%%%%%%%%%%
%Quelle zu Shape opt?
%Andere Ansätze als direkt Shape opt, zb splines, ffd etc
%Formfunktional erklären in einem kurzen Abschnitt?
%wie zitiere ich Schulz's Vorlesung?
%material- und formableitung?
%variationsungleichungen?
%%%%%%%%%%%%%%%%%%%%%%%%%%%%%%%%%%%

\section{Einführung in die Formoptimierung}

In diesem Kapitel möchten wir eine Einführung in die Theorie der Formoptimierung geben.
Die Formoptimierung ist der Bereich des Optimal Control, welcher sich mit der Wahl einer, in gewisser Hinsicht, optimalen Form beschäftigt. Anders als in der klassischen PDE-constrained Optimization ist hier die Steuerung somit keine Funktion, sondern eine Form.
Zu dem Umgang mit Optimierungsproblemen von Formen gibt es in der Literatur verschiedene Ansätze. \newline
Ein möglicher Ansatz zur Lösung ist die direkte Parametrisierung. Das zugrunde liegende Gebiet wird mit Hilfe von Kurven erzeugt, wobei diese von Parametern abhängig sind. Dadurch wird die Form mit Hilfe der endlich vielen Parameter gesteuert, nach denen im Anschluss optimiert werden kann. Der Vorteil dieser Methode ist, dass man sich in die Situation des Optimierens in endlichdimensionalen Vektorräumen zurückziehen kann, was erhebliche theoretische und numerische Erleichterungen mit sich bringt. Beispielhaft sei hier \cite{b-spline} genannt, wo die Autoren sogenannte B-Splines verwenden, um mit Hilfe eines adaptiv knotenerzeugenden Algorithmus in Kombination mit dem BFGS-Verfahren verbesserte Konvergenz zu erzielen. Ein großer Nachteil dieses Ansatzes ist, dass nicht beliebige Geometrien erreicht werden können, da diese wesentlich von der Wahl der parametrisierenden Kurven abhängt. \newline
Ein weiterer Ansatz, den wir in dieser Arbeit weiter verfolgen werden, ist die Verwendung des sogenannten \textit{Formkalküls} (engl. \textit{shape calculus}), siehe beispielsweise \cite{Shape_diff} und \cite{shapeopt}. Hierbei wird die Form nicht endlich parametrisiert, stattdessen findet die Optimierung im Raum aller Formen (engl. \textit{shape space}) statt. Dies bietet den Vorteil, dass sich dieser Ansatz nicht künstlich restriktiv auf die Auswahl möglicher Geometrien auswirkt. Als Hindernisse bei diesem Ansatz treten eine Vielzahl von Phänomenen auf. Beispielsweise weist der Raum aller Formen keine natürliche Vektorraumstruktur auf, weshalb die Optimierung in diesem Raum sich nicht auf Optimierung in Banach- oder Hilberträumen zurückführen lässt. Somit stellt sich die Frage nach einer sinnvollen Wahl der Metrik auf diesem Raum, sowie dem Umgang mit den so erzeugten Strukturen und deren numerische Ausnutzung.
Um nicht den Rahmen dieser Arbeit zu sprengen, verweisen für dieses Gebiet auf \cite{riemann}, \cite{riemann2} und \cite{shape_space}, sowie der  \colorbox{red}{Vorlesung ...}, und bedienen uns stattdessen lediglich der grundlegenden Begrifflichkeiten und einiger ausgewählter Konzepte. In diese möchten wir, weitestgehend bei den Grundlagen beginnend, einführen und beginnen mit den Begriffen zur Definition von Formoptimierungsproblemen.

Bei der numerischen Optimierung werden Objekte hinsichtlich durch Zielfunktionale definierter Kriterien bewertet. Die Güte einer Form ist dabei, siehe \cite{shape_space}, gegeben durch den skalaren Wert bei Auswertung eines sogenannten Formfunktionals.

\begin{defi}[Formfunktional] %\label{Formfunktional}
Sei $\mathcal{O}\subset \mathbb{R}^d$ nicht leer und offen mit $d \in \mathbb{N}_+$. Weiterhin sei $\mathcal{D} \subseteq 2^{\mathcal{O}}$ die Menge aller abgeschlossenen Teilmengen von $\mathcal{O}$. Dann heißt die Abbildung 
\begin{align*}
\mathcal{J}: \mathcal{D} \rightarrow \mathbb{R},\; \Omega \rightarrow \mathcal{J}(\Omega)
\end{align*}
\textit{Formfunktional} und $\Omega \in \mathcal{D}$ \textit{zulässiges Gebiet}.
\end{defi}

In der Literatur wird die Menge $\mathcal{O}$ oft \textit{hold-all-domain} genannt. Alle Formen, die zur Optimierung in Betracht gezogen werden, lassen sich in die hold-all-domain einbetten. Abgeschlossenheit wird gefordert, um bei jeder Form $\Omega$ die Existenz eines sinnvollen Randes $\partial\Omega$ zu sichern. 

Ziel der Formoptimierung ist es nun, ein Optimierungsproblem eines Formfunktionals mit Nebenbedingungen zu lösen.

\textcolor{red}{(...)}

Da die Menge aller zur Optimierung verwendeten Formen $\mathcal{D}$ aus Perspektive der Praxis zu groß ist, fordern wir einige Regularitätseigenschaften, in Anlehnung an \cite{Shape_diff} und \colorbox{red}{schulz vorlesung}:

\begin{defi}[reguläre Formen]\label{regu}
Sei $\Omega \in \mathcal{D}$ ein zulässiges Gebiet. Falls gilt:
\begin{itemize}
\item[i)] $\partial \Omega$ ist eine (d-1)-dimensionale Untermannigfaltigket des $\mathbb{R}^d$, und in jedem Punkt $p\in \partial \Omega$ offene Umgebungen $U\subseteq \mathbb{R}^{d-1}, V\subseteq \mathbb{R}^d$ mit einer zugehörigen Parametrisierung $\varphi: U \rightarrow \mathbb{R}^d$ und einer hinreichend differenzierbaren Erweiterung $h: V \rightarrow \mathbb{R}^d$ existieren, so dass lokal gilt:\\
$h(x,0) = \varphi(x)$ und $h(x,x_d) \in int(\Omega)$ falls $-\varepsilon < x_d < 0$, wobei $\varepsilon > 0$ und $int(\Omega)$ das Innere von $\Omega$ ist.
\item[i)] $\Omega$ ist zusammenhängend, beschränkt und besitzt einen Lipschitzrand.
\end{itemize}
Dann heißt $\Omega$ reguläres Gebiet.
\end{defi}

Anschaulich bedeutet die erste Forderung an Regularität, dass sich $\partial\Omega$ lokal so parametrisieren lässt, dass durch hinzufügen einer weiteren Koordinate $x_d$ der umgebende Raum mit parametrisiert wird und anhand des Vorzeichens von $x_d$ erkennbar ist, ob sich der zugehörige Punkt $h(x,x_d)$ im Inneren oder außerhalb von $\Omega$ befindet.

Nun möchten wir die Problemstellung der Formoptimierung im Rahmen dieser Arbeit definieren, in Analogie zu \colorbox{red}{Vorlesung schulz}:

\begin{defi}[Problem der Formoptimierung]
Sei $\mathcal{D}$ die Menge aller zulässigen Gebiete einer hold-all-domain $\mathcal{O}\subseteq\mathbb{R}^d$ und $\mathcal{J}$ ein auf dieser Menge beschränktes, wohldefiniertes Formfunktional. Weiterhin seien $Y,Z$ Banachräume und $c: Y\times \mathcal{D} \rightarrow Z$ eine Abbildung. Dann heißt das Problem

\begin{align*}
	\underset{(y,\Omega) \in Y \times \mathcal{D}}{\min} \mathcal{J}(y,\Omega) \\
	\text{unter } c(y, \Omega) = 0
\end{align*}
\textit{Problem der Formoptimierung}.
\end{defi}

Die Nebenbedingung $c(y, \Omega) = 0$ wird in dieser Arbeit eine Partielle Differentialgleichung auf dem Gebiet $\Omega$ mit Lösung $y$ sein. Da $\mathcal{D}$ wie zuvor erwähnt keine Vektorraumstruktur besitzt, lässt sich $\mathcal{D}$ im Allgemeinen nicht als Hilbert- oder Banachraum auffassen, wodurch Formoptimierungsprobleme in dieser Formulierung aus strukureller Sicht im Allgemeinen deutlich komplexer sind als Optimierungsprobleme in Hilbert- oder Banachräumen. Für die Grundlagen dieser Theorie der Formoptimierung verweisen wir auf \cite{shape_space}.

Der in Definition \ref{regu} geforderte Lipschitzrand ist nun sowohl aus theoretischer, als aus praktischer Sicht zu rechtfertigen, da Existenz und Eindeutigkeit der im Anschluss betrachteten PDE auf Gebieten mit Lipschitzrand gesichert sind, und diese somit für die Wohlgestelltheit des Optimierungsproblems notwendig sind. 

Die Verfahren zur Optimierung von Formen bei partiellen Differentialgleichungen, die wir in dieser Arbeit betrachten, basieren auf der Differenzierbarkeit der obigen Ausdrücke, unter anderem bezüglich der Form $\Omega$. Im Folgenden führen wir die hierzu gehörigen Begriffe ein. Zunächst betrachten wir Deformationen von Gebieten in Analogie zu \cite{bfgs2}.

\begin{defi}[Deformiertes Gebiet]
Sei $\Omega\in\mathcal{D}$ ein zulässiges Gebiet, $\tau > 0$, sei $\{T_t\}_{t\in [0,\tau]}$ eine Familie von bijektiven Abbildungen $T_t: \Omega \rightarrow \mathbb{R}^d$ mit $T_0 = id$. Dann heißt $\Omega_t := \{T_t(x) \vert x\in\Omega\}$ \textit{deformiertes Gebiet}.
\end{defi}

Zwei häufig Verwendung findende Arten solcher Familien von Deformationen $\{T_t\}_{t\in [0,\tau]}$ sind die \textit{Pertubation of Identity} und die \textit{Velocity method}, siehe \cite{bfgs2}. Die \textit{Pertubation of Identity} ist definiert über ein hinreichend differenzierbares Vektorfeld $V \in C^2(\mathbb{R}^d,\mathbb{R}^d)$ durch
\begin{align*}
	T_t: \Omega \rightarrow \mathbb{R}^d, x \mapsto x + tV(x).
\end{align*}
Die Deformationen der \textit{Velocity method} sind definiert durch den Fluss $T_t(x) := \xi(t,x)$, welcher gegeben ist durch das Anfangswertproblem
\begin{align*}
	\frac{d\xi(t,x)}{dt} &= V(\xi(t,x))\\
	\xi(0,x) 		&= x.
\end{align*}

Für den Rest dieser Arbeit werden wir die Pertubation of Identity verwenden. Mit Hilfe der eingeführten Deformationen können wir nun die Formableitung nach \cite{bfgs2} definieren. 

\begin{defi}[Formableitung]
Sei $\Omega \in \mathcal{D}$ eine zulässige Form, $V\in C^2(\mathbb{R}^d,\mathbb{R}^d)$ ein differenzierbares Vektorfeld, und $\mathcal{J}: \mathcal{D} \rightarrow \mathbb{R}$ ein Formfunktional. Falls der Grenzwert
\begin{align*}
	d\mathcal{J}(\Omega)[V] := \underset{t \rightarrow 0^+}{\lim}
	\frac{\mathcal{J}(\Omega_t) - \mathcal{J}(\Omega)}{t}
\end{align*}
existiert, und die Abbildung 
\begin{align*}
	d\mathcal{J}(\Omega)[\cdot]: C^2(\mathbb{R}^d,\mathbb{R}^d) \rightarrow 						\mathbb{R}, V \mapsto d\mathcal{J}(\Omega)[V]
\end{align*}
stetig und linear für alle $V \in C^2(\mathbb{R}^d,\mathbb{R}^d)$ ist, so heißt $d\mathcal{J}(\Omega)[V]$ \textit{Formableitung} von $\mathcal{J}$ in $\Omega$ in Richtung $V$.
\end{defi}

Die Formableitung besitzt zwei äquivalente Formulierungen, die sogenannte \textit{Volumenformulierung}, gekennzeichnet durch $\Omega$, und die \textit{Randformulierung}, gekennzeichnet durch $\Gamma := \partial \Omega$, siehe \cite{bfgs2}:

\begin{align*}
	dJ_\Omega(\Omega)[V] &= \underset{\Omega}{\int}F(x)V(x)dx\\
	\vspace{0.5cm}
	dJ_\Gamma(\Omega)[V] &= \underset{\Gamma}{\int}f(s)V(s)^Tn(s)ds,
\end{align*}

wobei $F(x)$ ein (differential) Operator ist, welcher linear auf das Richtungsvektorfeld $V$ wirkt, sowie einer Funktion $f: \Gamma \rightarrow \mathbb{R}$. Die Äquivalenz dieser Formulierungen ist gesichert durch den Hadamard'schen Darstellungssatz, wobei der dortige Zusammenhang geringfügig komplexer ist.

Die hier definierte Formableitung ist im Englischen auch bekannt als \textit{Eulerian Semiderivative} oder \textit{Lie Semiderivative}. Es gibt weitere äquivalente Varianten die Formableitung zu definieren, hierzu siehe \cite{Shape_diff}. Oft von zentraler Schwierigkeit in der theoretischen Behandlung von Formoptimierungsproblemen ist der Nachweis der Formdifferenzierbarkeit eines Formfunktionals. Hierzu gibt es eine Vielzahl von verschiedenen Techniken, beispielsweise der Min-Max Formulierung von Correa und Seeger, Céa's klassischer Lagrange-Methode oder der Methoden mittels Materialableitung. Erneut verweisen wir auf \cite{Shape_diff}, wo diese Methoden in einheitlicher Notation geordnet zusammengetragen sind.