\documentclass[bibliography=totoc,12pt,a4paper]{scrartcl}
\usepackage{amsmath, amssymb, amsthm}
\usepackage{enumerate}% schicke Nummerierung
\usepackage{graphicx}
\usepackage[english, ngerman]{babel}
\usepackage[T1]{fontenc}
\usepackage{lmodern}
\usepackage[utf8]{inputenc}
\usepackage{bigdelim}
\usepackage{multirow}
\usepackage{dsfont}
\usepackage[colorlinks=true,linkcolor=black, citecolor=black]{hyperref}
\usepackage{cite}
\usepackage[nottoc]{tocbibind}
\usepackage{empheq}
\usepackage{fancyhdr}
\usepackage{geometry}
\usepackage{lipsum}
\usepackage{tikz,pgfplots}
\usepackage{nicefrac}
\usetikzlibrary{shapes.misc}
\usetikzlibrary{matrix}
\geometry{a4paper,left=40mm,right=30mm, top=5cm, bottom=5cm} 

\def\@biblabel#1{\textcolor{red}{[#1]}}

\newtheoremstyle{linebreak}   % name
{3pt}                         % Space above
{3pt}                         % Space below
{}                            % Body font
{}                            % Indent amount 1
{\bfseries}                   % Theorem head font
{\newline}                    % Punctuation after theorem head
{.5em}                        % Space after theorem head 2
{}                            % Theorem head spec (can be left empty, meaning ‘normal’)
%\theoremstyle{linebreak}
\newtheoremstyle{exampstyle}
  {\topsep} % Space above
  {\topsep} % Space below
  {} % Body font
  {} % Indent amount
  {\bfseries} % Theorem head font
  {.} % Punctuation after theorem head
  {.5em} % Space after theorem head
  {} % Theorem head spec (can be left empty, meaning `normal')
\theoremstyle{exampstyle}
\newtheorem{defi}{Definition}%[chapter]
\newtheorem{satz}[defi]{Satz}
\newtheorem{theorem}[defi]{Theorem}
\newtheorem{propo}[defi]{Proposition}
\newtheorem{lemma}[defi]{Lemma}
\newtheorem{cor}[defi]{Korollar}
\newtheorem{bem}[defi]{Bemerkung}
\newtheorem{bsp}[defi]{Beispiel}
\newtheorem{folg}[defi]{Folgerung}
%bemerkungen oder Fließtext???
\numberwithin{equation}{section} 
 \newcommand{\newln}{\\&\quad\quad{}}
 \setlength\parindent{0pt}

\renewenvironment{abstract}
 {\small
  \begin{center}
  \bfseries \abstractname\vspace{-.5em}\vspace{0pt}
  \end{center}
  \list{}{%
    \setlength{\leftmargin}{12mm}% <---------- CHANGE HERE
    \setlength{\rightmargin}{\leftmargin}%
  }%
  \item\relax}
 {\endlist}




\begin{document}

\title{Quasi-Newton Methoden in der Formoptimierung}

\author{Daniel Luft \\ Prof. Dr. V. Schulz}

  \pagestyle{empty}

  % Titelblatt der Arbeit
  \begin{titlepage}

%    \includegraphics[scale=0.45]{kit-logo.jpg}
    \vspace*{2cm} 

 \begin{center} \large 
    
    Masterarbeit
    %Titel des Seminars
    \vspace*{2cm}

    {\huge Quasi-Newton Methoden in der Formoptimierung}
    \vspace*{2.5cm}

    Daniel Luft
    \vspace*{6cm}


	Universität Trier
  \end{center}
\end{titlepage}
%\maketitle

  % Inhaltsverzeichnis
  \tableofcontents

\newpage

  % Ab sofort Seitenzahlen in der Kopfzeile anzeigen
  \pagestyle{headings}
  
\selectlanguage{ngerman}
\begin{abstract}
Abstract
\end{abstract}
\vspace{1cm}



%%%%%%%%%%%%%%%%%%%%%%%%%%%%%%%%%%%
%Quelle zu Shape opt?
%Andere Ansätze als direkt Shape opt, zb splines, ffd etc
%Formfunktional erklären in einem kurzen Abschnitt?
%wie zitiere ich Schulz's Vorlesung?
%material- und formableitung?
%variationsungleichungen?
%%%%%%%%%%%%%%%%%%%%%%%%%%%%%%%%%%%

\section{Einführung in die Formoptimierung}

\colorbox{red}{zuerst pde kapitel!}
\colorbox{red}{überlege, ob man vllt grundgebiet anders nennt, und dann formen partialOmega statt partialOmega2 nennt... vllt einfacher Gamma int und Omega int}
\colorbox{red}{ sollte man Zieltfktnl J als Abbildung der formen partial Omega oder der Inneren der Formen Omega auffassen, eigentlich egal, aber was wäre schöner? achte auf konsistenz der sprache, z.b. bei erklärung nach designgleichung}

In diesem Kapitel möchten wir eine Einführung in die Theorie der Formoptimierung geben.
Die Formoptimierung ist der Bereich des Optimal Control, welcher sich mit der Wahl einer, in gewisser Hinsicht, optimalen Form beschäftigt. Anders als in der klassischen PDE-constrained Optimization ist hier die Steuerung somit keine Funktion, sondern eine Form.
Zu dem Umgang mit Optimierungsproblemen von Formen gibt es in der Literatur verschiedene Ansätze. \newline
Ein möglicher Ansatz zur Lösung ist die direkte Parametrisierung. Das zugrunde liegende Gebiet wird mit Hilfe von Kurven erzeugt, wobei diese von Parametern abhängig sind. Dadurch wird die Form mit Hilfe der endlich vielen Parameter gesteuert, nach denen im Anschluss optimiert werden kann. Der Vorteil dieser Methode ist, dass man sich in die Situation des Optimierens in endlichdimensionalen Vektorräumen zurückziehen kann, was erhebliche theoretische und numerische Erleichterungen mit sich bringt. Beispielhaft sei hier \cite{b-spline} genannt, wo die Autoren sogenannte B-Splines verwenden, um mit Hilfe eines adaptiv knotenerzeugenden Algorithmus in Kombination mit dem BFGS-Verfahren verbesserte Konvergenz zu erzielen. Ein großer Nachteil dieses Ansatzes ist, dass nicht beliebige Geometrien erreicht werden können, da diese wesentlich von der Wahl der parametrisierenden Kurven abhängt. \newline
Ein weiterer Ansatz, den wir in dieser Arbeit weiter verfolgen werden, ist die Verwendung des sogenannten \textit{Formkalküls} (engl. \textit{shape calculus}), siehe beispielsweise \cite{Shape_diff} und \cite{shapeopt}. Hierbei wird die Form nicht endlich parametrisiert, stattdessen findet die Optimierung im Raum aller Formen (engl. \textit{shape space}) statt. Dies bietet den Vorteil, dass sich dieser Ansatz nicht künstlich restriktiv auf die Auswahl möglicher Geometrien auswirkt. Als Hindernisse bei diesem Ansatz treten eine Vielzahl von Phänomenen auf. Beispielsweise weist der Raum aller Formen keine natürliche Vektorraumstruktur auf, weshalb die Optimierung in diesem Raum sich nicht auf Optimierung in Banach- oder Hilberträumen zurückführen lässt. Somit stellt sich die Frage nach einer sinnvollen Wahl der Metrik auf diesem Raum, sowie dem Umgang mit den so erzeugten Strukturen und deren numerische Ausnutzung.
Um nicht den Rahmen dieser Arbeit zu sprengen, verweisen für dieses Gebiet auf \cite{riemann}, \cite{riemann2} und \cite{shape_space}, sowie der  \colorbox{red}{Vorlesung ...}, und bedienen uns stattdessen lediglich der grundlegenden Begrifflichkeiten und einiger ausgewählter Konzepte. In diese möchten wir, weitestgehend bei den Grundlagen beginnend, einführen und beginnen mit den Begriffen zur Definition von Formoptimierungsproblemen.

Bei der numerischen Optimierung werden Objekte hinsichtlich durch Zielfunktionale definierter Kriterien bewertet. Die Güte einer Form ist dabei, siehe \cite{shape_space}, gegeben durch den skalaren Wert bei Auswertung eines sogenannten Formfunktionals.

\begin{defi}[Formfunktional] %\label{Formfunktional}
Sei $\mathcal{O}\subset \mathbb{R}^d$ nicht leer und offen mit $d \in \mathbb{N}_+$. Weiterhin sei $\mathcal{D} \subseteq 2^{\mathcal{O}}$ die Menge aller abgeschlossenen Teilmengen von $\mathcal{O}$. Dann heißt die Abbildung 
\begin{align*}
\mathcal{J}: \mathcal{D} \rightarrow \mathbb{R},\; \Omega \rightarrow \mathcal{J}(\Omega)
\end{align*}
\textit{Formfunktional} und $\Omega \in \mathcal{D}$ \textit{zulässiges Gebiet}.
\end{defi}

In der Literatur wird die Menge $\mathcal{O}$ oft \textit{hold-all-domain} genannt. Alle Formen, die zur Optimierung in Betracht gezogen werden, lassen sich in die hold-all-domain einbetten. Abgeschlossenheit wird gefordert, um bei jeder Form $\Omega$ die Existenz eines sinnvollen Randes $\partial\Omega$ zu sichern. 

Ziel der Formoptimierung ist es nun, ein Optimierungsproblem eines Formfunktionals mit Nebenbedingungen zu lösen.

\textcolor{red}{(...)}

Da die Menge aller zur Optimierung verwendeten Formen $\mathcal{D}$ aus Perspektive der Praxis zu groß ist, fordern wir einige Regularitätseigenschaften, in Anlehnung an \cite{Shape_diff} und \colorbox{red}{schulz vorlesung}:

\begin{defi}[reguläre Formen]\label{regu}
Sei $\Omega \in \mathcal{D}$ ein zulässiges Gebiet. Falls gilt:
\begin{itemize}
\item[i)] $\partial \Omega$ ist eine (d-1)-dimensionale Untermannigfaltigket des $\mathbb{R}^d$, und in jedem Punkt $p\in \partial \Omega$ offene Umgebungen $U\subseteq \mathbb{R}^{d-1}, V\subseteq \mathbb{R}^d$ mit einer zugehörigen Parametrisierung $\varphi: U \rightarrow \mathbb{R}^d$ und einer hinreichend differenzierbaren Erweiterung $h: V \rightarrow \mathbb{R}^d$ existieren, so dass lokal gilt:\\
$h(x,0) = \varphi(x)$ und $h(x,x_d) \in int(\Omega)$ falls $-\varepsilon < x_d < 0$, wobei $\varepsilon > 0$ und $int(\Omega)$ das Innere von $\Omega$ ist.
\item[i)] $\Omega$ ist zusammenhängend, beschränkt und besitzt einen Lipschitzrand.
\end{itemize}
Dann heißt $\Omega$ reguläres Gebiet.
\end{defi}

Anschaulich bedeutet die erste Forderung an Regularität, dass sich $\partial\Omega$ lokal so parametrisieren lässt, dass durch hinzufügen einer weiteren Koordinate $x_d$ der umgebende Raum mit parametrisiert wird und anhand des Vorzeichens von $x_d$ erkennbar ist, ob sich der zugehörige Punkt $h(x,x_d)$ im Inneren oder außerhalb von $\Omega$ befindet.

Nun möchten wir die Problemstellung der Formoptimierung im Rahmen dieser Arbeit definieren, in Analogie zu \cite{LagrangeNewton}.

\colorbox{red}{sollte ich das so nennen? in Banachräumen}
\begin{defi}[Abstraktes Formoptimierungsproblem]
Sei $\mathcal{D}$ die Menge aller zulässigen Gebiete einer hold-all-domain $\mathcal{O}\subseteq\mathbb{R}^d$ und $\mathcal{J}$ ein auf dieser Menge beschränktes, wohldefiniertes Formfunktional. Weiterhin seien $Y,Z$ Banachräume und $c: Y\times \mathcal{D} \rightarrow Z$ eine hinreichend glatte Abbildung. Dann heißt das Problem

\begin{align*}
	\underset{(y,\Omega) \in Y \times \mathcal{D}}{\min} \mathcal{J}(y,\Omega) \\
	\text{s.t. } c(y, \Omega) = 0
\end{align*}
\textit{Problem der Formoptimierung}.
\end{defi}

Die Nebenbedingung $c(y, \Omega) = 0$ wird in dieser Arbeit eine Partielle Differentialgleichung auf dem Gebiet $\Omega$ mit Lösung $y$ sein. Da $\mathcal{D}$ wie zuvor erwähnt keine Vektorraumstruktur besitzt, lässt sich $\mathcal{D}$ im Allgemeinen nicht als Hilbert- oder Banachraum auffassen, wodurch Formoptimierungsprobleme in dieser Formulierung aus strukureller Sicht im Allgemeinen deutlich komplexer sind als Optimierungsprobleme in Hilbert- oder Banachräumen. Für die Grundlagen dieser Theorie der Formoptimierung verweisen wir auf \cite{shape_space}.

Der in Definition \ref{regu} geforderte Lipschitzrand ist nun sowohl aus theoretischer, als aus praktischer Sicht zu rechtfertigen, da Existenz und Eindeutigkeit der im Anschluss betrachteten PDE auf Gebieten mit Lipschitzrand gesichert sind, und diese somit für die Wohlgestelltheit des Optimierungsproblems notwendig sind. 

Die Verfahren zur Optimierung von Formen bei partiellen Differentialgleichungen, die wir in dieser Arbeit betrachten, basieren auf der Differenzierbarkeit der obigen Ausdrücke, unter anderem bezüglich der Form $\Omega$. Im Folgenden führen wir die hierzu gehörigen Begriffe ein. Zunächst betrachten wir Deformationen von Gebieten in Analogie zu \cite{bfgs2}.

\begin{defi}[Deformiertes Gebiet]
Sei $\Omega\in\mathcal{D}$ ein zulässiges Gebiet, $\tau > 0$, sei $\{T_t\}_{t\in [0,\tau]}$ eine Familie von bijektiven Abbildungen $T_t: \Omega \rightarrow \mathbb{R}^d$ mit $T_0 = id$. Dann heißt $\Omega_t := \{T_t(x) \vert x\in\Omega\}$ \textit{deformiertes Gebiet}.
\end{defi}

Zwei häufig Verwendung findende Arten solcher Familien von Deformationen $\{T_t\}_{t\in [0,\tau]}$ sind die \textit{Pertubation of Identity} und die \textit{Velocity method}, siehe \cite{bfgs2}. Die \textit{Pertubation of Identity} ist definiert über ein hinreichend differenzierbares Vektorfeld $V \in C^2(\mathbb{R}^d,\mathbb{R}^d)$ durch
\begin{align*}
	T_t: \Omega \rightarrow \mathbb{R}^d, x \mapsto x + tV(x).
\end{align*}
Die Deformationen der \textit{Velocity method} sind definiert durch den Fluss $T_t(x) := \xi(t,x)$, welcher gegeben ist durch das Anfangswertproblem
\begin{align*}
	\frac{d\xi(t,x)}{dt} &= V(\xi(t,x))\\
	\xi(0,x) 		&= x.
\end{align*}

Für den Rest dieser Arbeit werden wir die Pertubation of Identity verwenden. Mit Hilfe der eingeführten Deformationen können wir nun die Formableitung nach \cite{bfgs2} definieren. 

\colorbox{red}{Eigenschaften der Formableitung? z.b. linearität etc, oder bei materialableitung? falls überhaupt, siehe design gleichung, da benutzen wir linearität}

\begin{defi}[Formableitung]
Sei $\Omega \in \mathcal{D}$ eine zulässige Form, $V\in C^2(\mathbb{R}^d,\mathbb{R}^d)$ ein differenzierbares Vektorfeld, und $\mathcal{J}: \mathcal{D} \rightarrow \mathbb{R}$ ein Formfunktional. Falls der Grenzwert
\begin{align*}
	d\mathcal{J}(\Omega)[V] := \underset{t \rightarrow 0^+}{\lim}
	\frac{\mathcal{J}(\Omega_t) - \mathcal{J}(\Omega)}{t}
\end{align*}
existiert, und die Abbildung 
\begin{align*}
	d\mathcal{J}(\Omega)[\cdot]: C^2(\mathbb{R}^d,\mathbb{R}^d) \rightarrow 						\mathbb{R}, V \mapsto d\mathcal{J}(\Omega)[V]
\end{align*}
stetig und linear für alle $V \in C^2(\mathbb{R}^d,\mathbb{R}^d)$ ist, so heißt $d\mathcal{J}(\Omega)[V]$ \textit{Formableitung} von $\mathcal{J}$ in $\Omega$ in Richtung $V$.
\end{defi}

Die Formableitung besitzt zwei äquivalente Formulierungen, die sogenannte \textit{Volumenformulierung}, gekennzeichnet durch $\Omega$, und die \textit{Randformulierung}, gekennzeichnet durch $\Gamma := \partial \Omega$, siehe \cite{bfgs2}:

\begin{align*}
	dJ_\Omega(\Omega)[V] &= \underset{\Omega}{\int}F(x)V(x)dx\\
	\vspace{0.5cm}
	dJ_\Gamma(\Omega)[V] &= \underset{\Gamma}{\int}f(s)\langle V(s),n(s)\rangle ds,
\end{align*}

wobei $F(x)$ ein (differential) Operator ist, welcher linear auf das Richtungsvektorfeld $V$ wirkt, sowie einer Funktion $f: \Gamma \rightarrow \mathbb{R}$. Die Existenz einer zur Volumenform äquivalenten Randformulierung ist gesichert durch den Hadamard'schen Darstellungssatz. Dieser liefert im Allgemeinen die Existenz einer Distribution $f$, so dass Volumen- und Randformulierung der  Formableitung äquivalent sind, für die genaue Aussage, siehe \cite{shape_space}, Theorem 4.7. Wir werden in dieser Arbeit stets voraussetzen, dass genügend Regularität vorhanden ist, so dass $dJ$ in der Tat mittels einer Funktion $f$ auf dem Rand darstellbar ist.

\colorbox{red}{ dJ oder dmathcal J??? konsistenz checken}

Die hier definierte Formableitung ist im Englischen auch bekannt als \textit{Eulerian Semiderivative} oder \textit{Lie Semiderivative}. Es gibt weitere äquivalente Varianten die Formableitung zu definieren, hierzu siehe \cite{Shape_diff}. Oft von zentraler Schwierigkeit in der theoretischen Behandlung von Formoptimierungsproblemen ist der Nachweis der Formdifferenzierbarkeit eines Formfunktionals. Hierzu gibt es eine Vielzahl von verschiedenen Techniken, beispielsweise der Min-Max Formulierung von Correa und Seeger, Céa's klassischer Lagrange-Methode oder der Methoden mittels Materialableitung. Erneut verweisen wir auf \cite{Shape_diff}, wo diese Methoden in einheitlicher Notation geordnet zusammengetragen sind.

\colorbox{red}{Rechenregeln? Materialableitung?}

Wir kommen nun zu einer, zwar immernoch abstrakt gehaltenen, jedoch konkreter anwendungsbezogenen Formulierung von Formoptimierungsproblemen.

\colorbox{red}{mit Bündeldefinition oder ohne? siehe paper lagrange-newton, außerdem im widerspruch zur definition eines formfunktionals oben??}
\begin{defi}[PDE beschränktes Formoptimierungsproblem]\label{Pde constrained shape}
Sei $\mathcal{J}$ ein Formfunktional, sei $\Omega_2 \in \mathcal{D}$ eine zulässige Fläche einer Hold-all-Domain $\mathcal{D}$. Weiterhin sei $y\in H(\Omega_2)$ eine Funktion aus dem von $\Omega_2$ abhängigen \colorbox{red}{Hilbertraum (oder Banachraum)} $H(\Omega_2)$. Betrachte eine von $\Omega_2$ abhängige Bilinearform $a_{\Omega_2}: H(\Omega_2) \times H(\Omega_2) \rightarrow \mathbb{R}$ und eine ebenso von $\Omega_2$ abhängige Linearform $b_{\Omega_2}: H(\Omega_2) \rightarrow \mathbb{R}$. Dann heißt das beschränkte Minimierungsproblem
\begin{equation}\label{PDE constrained equation}
	\begin{aligned}
	&\underset{y,\Omega_2}{\min}\;\mathcal{J}(y,\Omega_2) \\
	\text{s.t. } a_{\Omega_2}&(y,p) = b_{\Omega_2}(p) \quad \forall p\in H(\Omega_2)
	\end{aligned}
\end{equation}
\textit{PDE beschränktes Formoptimierungsproblem} (engl.\textit{PDE-constrained shape-optimization}).
\end{defi}

Wir haben hier direkt die partielle Differentialgleichung in Variationsformulierung als Gleichungsnebenbedingung formuliert. Das wird uns bequemer ermöglichen, weitere Definitionen und Techniken entsprechend anzugeben, womit wir direkt beginnen.

\begin{defi}[Lagrangefunktion]\label{lagrangefunction}
Betrachte ein PDE beschränktes Formoptimierungsproblem. Sei $\partial\Omega_2$ eine reguläre Form, $H(\Omega_2)$ ein von dieser Form abhängiger \colorbox{red}{Hilbertraum / Banachraum}. Weiterhin seien $y,p\in H(\Omega_2)$. Dann heißt die Funktion
\colorbox{red}{ich unterschlage hier mutwillig die bündelgeschichte, soll ich die noch aufführen? kann man, wird korrekter, aber länger}.
\begin{align*}
	\mathcal{L}(y,\Omega_2, p) := \mathcal{J}(y,\Omega_2) + a_{\Omega_2}(y,p) - b_{\Omega_2}(p)
\end{align*}
\textit{Lagrangefunktion} für das gegebene PDE beschränkte Formoptimierungsproblem.
\end{defi}

Diese Lagrangefunktion ermöglicht es uns, aus formaler Sicht Optimalitätskriterien für das PDE beschränkte Formoptimierungsproblem aufzustellen. Ein problematischer Aspekt des rein formalen Lagrangeansatzes ist, dass die Regularität der Funktionen, für welche die Lagrangefunktion definiert ist, im Vorfeld nicht bekannt sein muss. Aus diesem Grunde ist eine ex-post Analyse der zur konsistenten Beschreibung nötigen Funktionenräume wichtig. \colorbox{red}{vllt eine Quelle?}

Wir fahren fort mit notwendigen Optimalitätskriterien für Lösungen, welche gegeben werden in Form von Variationsformulierungen, siehe \cite{LagrangeNewton}

\begin{defi}[Adjungierte und Design-Gleichung]
	Betrachte ein PDE beschränktes Formoptimierungsproblem mit differenzierbarem Zielfunktional $\mathcal{J}$. Verwende die Notation von \ref{lagrangefunction} und bezeichne mit $D\mathcal{J}$ die zu $\mathcal{J}$ gehörige Formableitung. Seien $\tilde{y} \in H(\tilde{\Omega}_2)$ und $\tilde{\Omega}_2\in \mathcal{D}$ optimale Lösungen des Problems \ref{PDE constrained equation}. Dann gilt die Variationsgleichung
\begin{equation}\label{adjoint equation}
	a_{\tilde{\Omega}_2}(\tilde{y},z) = - \frac{\partial}{\partial y} \mathcal{J}(\tilde{y},\tilde{\Omega}_2)(z) \quad \forall z\in H(\tilde{\Omega}_2),
\end{equation}
welche wir als \textit{adjungierte Gleichung} bezeichnen. Sei $\tilde{p} \in H(\tilde{\Omega}_2)$ die Lösung der adjungierten Gleichung \ref{adjoint equation}. Dann gilt weiterhin die Variationsgleichung
\colorbox{red}{ relativ frei definiert, kriege ich das abgesegnet? Vielleicht lieber Lagrangefunktion?}
\begin{equation}\label{Design equation}
	D\left(\mathcal{J}(\tilde{y},\tilde{\Omega}_2) + a_{\tilde{\Omega}_2}(\tilde{y},\tilde{p}) + b_{\tilde{\Omega}_2}(\tilde{p})\right)(\tilde{\Omega}_2)[V] = 0 \quad \forall C^\infty(\tilde{\Omega}_2,\tilde{\Omega}_2),
\end{equation}
welche wir als \textit{Design Gleichung} bezeichnen werden.
\textit{Design Gleichung}. Außerdem gilt
\begin{equation}
	a_{\tilde{\Omega}_2}(\tilde{y},p) = b_{\tilde{\Omega}_2}(p) \quad \forall p \in H(\tilde{\Omega}_2),
\end{equation}
wobei die Gleichungsnebenbedingung aus \ref{PDE constrained equation} \textit{Zustandsgleichung} genannt wird.
\end{defi}

Wie man unschwer erkennt, entstehen die notwendigen Bedingungen aus den Ableitungen der Lagrangefunktion \ref{lagrangefunction} nach dem Zustand $y$, der \colorbox{red}{Form} $\partial\Omega_2$ und dem adjungierten Zustand $p$. Fasst man diese Bedingung in einer Gleichung zusammen, so erhält man das KKT-System
\begin{align*}
	D\mathcal{L}(\tilde{y},\tilde{\Omega}_2,\tilde{p})\left(
	\begin{matrix}
	h_y \\
	h_{\Omega} \\
	h_p
	\end{matrix}\right)	 = 0 \quad \forall h_{\Omega} \in C^{\infty}(\tilde{\Omega}_2,\tilde{\Omega}_2)
	\forall h_y, h_p \in H(\tilde{\Omega}_2)
\end{align*}
als notwenige Bedingung zur Lösung von \ref{PDE constrained equation}, siehe \cite{LagrangeNewton}, wobei wir die dortige Formulierung mit Hilfe von Vektorbündeln auf die hier angepasste Notation übersetzt haben, was aufgrund der Definition des Tangentialbündels mittels von Formen abhängiger Kreuzprodukte von Hilberträumen  möglich ist. An dieser Stelle wäre es möglich einen Lagrange-Newton Ansatz durchzuführen, und ein Verfahren zur Lösung des Problems \ref{PDE constrained equation} mit quadratischer Konvergenz zu gewinnen, was die Autoren von \cite{LagrangeNewton} getan haben. Wir sparen uns an dieser Stelle weitere Ausführungen hierzu, da wir uns auf Quasi-Newton Methoden konzentrieren wollen.

Wir führen jetzt noch eine technische Notation zur Beschreibung von Sprüngen auf Rändern ein, vgl. \cite{LagrangeNewton}, woraufhin wir unser Modellproblem definieren.

\colorbox{red}{ist das technisch überhaupt korrekt?? kann ja beides nicht auf dem Rand definiert sein wenn man es auf offene Gebiete Einschränkt, höchsten auf Abschlüssen... oder ecklige Grenzwerte}
\begin{defi}
	Sei $\Omega_2 \subset \Omega$ ein reguläres Gebiet. Dann definieren wir das \textit{Sprungsymbol} $[[\cdot]]$, für eine auf $\Omega$ definierte Funktion $f$, auf dem Rand $\partial\Omega_2$ durch
	\begin{align*}
		[[f]] = f_{\vert \Omega \setminus \Omega_2} - f_{\vert \Omega_2}.
	\end{align*}
\end{defi}

Wir besitzen nun den Apparat zur Formulierung unseres Modellproblems und zugehöriger Optimalitätsbedingungen und Ableitungen. Hierzu wählen wir das klassische Poissonproblem, \colorbox{red}{Referenz im anderen Kapitel}, für welches wir ausreichende Theorie in \colorbox{red}{xxx} eingeführt haben. Das Problem findet sich in ähnlicher Art in \cite{shape_space}, 4.2., sowie in \cite{Lagrange-Newton}, unter Remark 2.

\colorbox{red}{sollte ich von irgendwo zitieren? sollte ich (0,1)2 Omega nenne?}
\colorbox{red}{ wichtig: brauche ich Stetigkeitsbedingungen und zugehörige Notation? siehe Führ}
\begin{defi}[Modellproblem]
	Sei $(0,1)^2 \subset \mathbb{R}^2$ das Einheitsquadrat im $\mathbb{R}^2$.
	Weiterhin sei $\Omega_2 \subset (0,1)^2$ ein reguläres Gebiet mit zugehöriger 		Form $\partial\Omega_2$ aus einer Hold-all-Domain $\mathcal{D}$, und eine stückweise konstante Funktion $f \in \mathcal{L}^2((0,1)^2)$ von der Form
	\begin{align*}
		f(x) &= f_1 \in \mathbb{R} \quad \forall x \in (0,1)^2\setminus \Omega_2 \\
		f(x) &= f_2 \in \mathbb{R} \quad \forall x \in \Omega_2.
	\end{align*}
	Sei $\hat{y} \in \mathcal{L}^2((0,1)^2)$ und $\mu > 0$. Dann heißt das nun folgende Problem \textit{Modellproblem}:
	\begin{equation}\label{Modellproblem}
	\begin{aligned}
	\underset{\Omega_2\in \mathcal{D}}{\min}\; \mathcal{J}(\Omega_2) :&= \frac{1}{2}\underset{\Omega_2}{\int} (y - \hat{y})^2 dx + \mu\underset{\partial\Omega_2}{\int} 1 ds \\
	\text{s.t.} -\Delta y &= f \quad \text{in } \;\Omega_2 \\
	y &= 0  \quad \text{auf } \partial\Omega_2.
	\end{aligned}
	\end{equation}		
	
	Zudem definieren wir für \ref{Modellproblem} explizit eine sogenannte \textit{Interface condition} auf $\partial\Omega_2$
	\begin{equation}\label{Interfacecondition}
		[[y]] = 0 \quad \Big[\Big[\frac{\partial y}{\partial n}\Big]\Big] = 0 \quad \text{auf } \partial \Omega_2,
	\end{equation}
	wobei $n$ das äußere Normalenvektorfeld auf $\partial\Omega_2$ ist.	Zusammen erhalten wir die schwache Formulierung der Zustandsgleichung
	\begin{equation}
		\underset{\Omega_2}{\int} \nabla y^T \nabla p\; dx - \underset{\partial\Omega_2}{\int} \Big[\Big[ \frac{\partial y}{\partial n}p\Big]\Big]\;ds = \underset{\Omega_2}{\int} fp \;dx \quad \forall p \in H^1_0(\Omega_2).
	\end{equation}
	
	Weiterhin besitzt das Modellproblem die, nach \ref{adjoint equation} 				definierte, adjungierte Gleichung \colorbox{red}{formatierung ändern damit gut aussieht}
	\begin{align*}\label{Modelladjoint}
		-\Delta p &= - (y - \hat{y}) \quad \text{in } \Omega_2 \\
		p &= 0 \\
		[[p]] &= 0 \quad \text{auf } \partial\Omega_2 \\ 
		\Big[\Big[\frac{\partial p}{\partial n}p\Big]\Big] &= 0 .
	\end{align*}	 
\end{defi}

\colorbox{red}{eventuell auch mit notation aOmega bOmega?}

Wie wir sehen, besteht das Zielfunktional $\mathcal{J}$ aus zwei Komponenten. Der erste Summand bildet das eigentliche Funktional von Interesse, welches wir von nun an mit $\mathcal{J}_{target}$ bezeichnen werden, wobei die optimale Form $\partial\tilde{\Omega}_2$ möglichst einen Zustand $\tilde{y}$ erzeugen soll, welcher einem gegebenen Sollzustand $\hat{y}$ entspricht. Beispiele hierzu wären etwa, \colorbox{red}{nötig? zitat?} die Temperatursteuerung bei gegebener konstanter Solltemperaturverteilung. Der zweite Summand
\begin{align*}
	\mu\underset{\partial\Omega_2}{\int} 1 ds =: \mathcal{J}_{reg}(\Omega_2)
\end{align*}
wird gemeinhin als \textit{Perimeter-Regularisierung} bezeichnet, wobei $\mu > 0$. Diese hat den Nutzen, Regularität und somit Existenz und Eindeutigkeit des Modellproblems \ref{Modellproblem} zu gewährleisten. Ohne die Regularisierung wären beispielsweise entartete Formen, deren Rand als Funktion unendlicher Variation darstellbar sind, zugelassen, siehe zum Beispiel die angegebene Quelle bei \cite{Lagrange-Newton}, Remark 2. \colorbox{red}{Quelle, zb aus Modellproblem kapitel bei shapespace?}

Die Zustandsgleichung ist ein gewöhnliches Poisson-Problem mit Dirichlet-Randwerten, welches wir hier symbolisch in starker Form notiert haben. Die starke Notation repräsentiert hier also die Variationsformulierung des Poisson-Problems mit Dirichlet Randwerten, wobei gleiches für die adjungierte Gleichung gilt. Für die Herleitung der adjungierten Gleichung, welche nach \ref{adjoint equation} erfolgt, verweisen wir auf \cite{shape_space}, 4.2.1. Die Existenz einer Lösung Zustands- als auch adjungierte Gleichung sind gesichert durch \colorbox{red}{Lax-milgram ref}.

Die Interface condition \ref{Interfacecondition} sorgt bei uns dafür, dass der Zustand $y$ stetig vom inneren Gebiet $\Omega_2$ zum äußeren Gebiet $\Omega \setminus \Omega_2$ verläuft, sowie, dass der Fluss $\frac{\partial y }{\partial n}$ auch stetig ist. Ohne die Interface condition wäre dies im Allgemeinen nicht der Fall, da Funktion $f$ der rechten Seite der Zustandsgleichung auf genau diesen Gebieten jeweils konstant ist, und auf $\partial\Omega_2$ springt.

Damit wir Wohldefiniertheit von im späteren Kapitel \colorbox{red}{cite, vllt sogar direkt die Metrik?} definierten Metriken für Formen sicherstellen können, benötigen wir, dass $\frac{\partial y}{\partial n} \in H^{1/2}(\partial\Omega_2)$ ist. Mit unseren Voraussetzungen $f,\bar{y} \in \mathcal{L}^2((0,1)^2)$, sowie einem hinreichend glatten Rand $\partial\Omega_2$, erhalten wir $y\in H^2((0,1)^2)$ nach \colorbox{red}{ref Regularität für Lösungen}. Kombiniert man dies mit den auf Sobolev-Slobodeckij-Räumen verallgemeinerten Spursatz \colorbox{red}{REFERENZ}, so erhalten wir wie gewünscht $\frac{\partial y}{\partial n} \in H^{1/2}(\partial\Omega_2)$. \colorbox{red}{fraglich, ob Ränder glatt?? und nicht sobolev?}

Wir können nun für unser Modellproblem \ref{Modellproblem} die Formableitung angeben. Die genaue Herleitung, die das Theorem von Correa-Seger verwendet und auf  Materialableitungen aufbaut, findet sich in \cite{shape_space}.

\begin{theorem}[Formableitung für das Modellproblem]
	Betrachte das Modellproblem \ref{Modellproblem} mit Interface condition. Sei $y\in H^1_0((0,1)^2)$ Lösung der Zustandsgleichung aus \ref{Modellproblem} und $p\in H^1_0((0,1)^2)$ Lösung der adjungierten Gleichung \ref{Modelladjoint}. Dann hat das Zielfunktional aus \ref{Modellproblem} \textit{ohne} Perimeter-Regularisierung $\mathcal{J}_{target}$ folgende Volumen-Formulierung für alle Formen $\partial\Omega_2 \in \mathcal{D}$ und Richtungen $V \in C^\infty(\Omega_2)$:
	\begin{equation}
	\begin{aligned}
		D\mathcal{J}_{target}(\Omega_2)[V] = \underset{\Omega_2}{\int} -\nabla y^T (\nabla V + \nabla V^T) \nabla p - p V^T \nabla f \\ + \text{div} (V) (\frac{1}{2}(y-\bar{y})^2 + \nabla y^T \nabla p - fp)\; dx.
	\end{aligned}
	\end{equation}
	Falls sogar genügend Regularität vorhanden ist, so dass $y \in H^2((0,1)^2)$ und $p \in H^2((0,1)^2)$, so besitzt die Formableitung die Randformulierung
	\begin{equation}
	D\mathcal{J}_{target}(\Omega_2)[V] = -\underset{\partial\Omega_2}{\int} [[f]]p \langle V,n \rangle ds.
	\end{equation}
	Zusammen mit der Perimeter-Regularisierung erhält man
	\begin{equation}
	D\mathcal{J}_{target}(\Omega_2)[V] = \underset{\partial\Omega_2}{\int} (-[[f]]p  + \kappa )\langle V,n \rangle ds,
	\end{equation}
	wobei $\kappa := \text{div}_{\partial\Omega_2}(n)$ die mittlere Krümmung von $\partial\Omega_2$ ist. \colorbox{red}{kappa so lassen?}
\end{theorem}

Diese Formableitungen lassen sich nun in ableitungsbasierten Optimierungsmethoden verwenden. Wie man sieht, muss man zum Auswerten der Formableitung in einer bestimmten Form $\partial\Omega_2$ die zugehörige Zustandsgleichung bei \ref{Modellproblem}, und die entsprechende adjungierte Gleichung \ref{adjoint equation} lösen. Weiter ließe sich dann aus der Ableitungsinformation ein Gradient erzeugen, mit dessen Hilfe man beispielsweise ein Verfahren des steilsten Abstiegs einsetzen kann. Um einen solchen Gradienten zu definieren, werden wir uns im folgenden Kapitel Gedanken machen, welche zugrundeliegende Metrik wir auf dem Raum aller Formen annehmen, um mit dieser einen Gradienten in Anlehnung an den Riesz'schen Darstellungssatzes zu definieren. An dieser Stelle sei erwähnt, dass wir hier Gebrauch von der geschlossenen Darstellung der Formableitung machen, welche wir in späteren Kapiteln implementieren werden. 

Ist die Formableitung bei einem gegebenen Problem nicht geschlossen angegeben, so lässt sich diese häufig numerisch mittels Techniken des \textit{automatischen Differenzierens} gewinnen. Diese Methoden, die eher aus der Numerik für Differentialgleichungen bekannt sind, lassen sich auch auf Formableitungen verallgemeinern. Dies tut beispielsweise der Autor von \cite{auto-diff}, um auf diesem Wege Hesse-Matrizen zu erzeugen. Hierzu wird die sogenannte \textit{Unified Form Language (UFL)} in der Programmiersprache Python verwendet, welche auch wesentlicher Bestandteil des Programmpackets FEniCS ist, für Details, siehe etwa die Quellen bei \cite{fenics}, S.25. 
\colorbox{red}{muss ich volumenform mit perimeter coden?}
\colorbox{red}{Übergang}




\newpage
\appendix
\section{Eidesstattliche Erklärung}
Hiermit erkläre ich, dass ich die Masterarbeit selbständig verfasst und keine anderen als die angegebenen Quellen und Hilfsmittel benutzt und die aus fremden Quellen direkt oder indirekt übernommenen
Gedanken als solche kenntlich gemacht habe. Die Masterarbeit habe ich bisher keinem anderen Prüfungsamt in gleicher oder vergleichbarer Form vorgelegt. Sie wurde bisher auch nicht veröffentlicht.
\vspace{1.5cm}

(Datum) \hspace{9cm} (Daniel Luft)


\newpage
\nocite{*}
\bibliographystyle{plain}
\bibliography{papers}

\end{document}